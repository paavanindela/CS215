\title{Assignment 1: CS 215}
\author{}
\date{Due: 3rd September before 11:55 pm, 100 points}

\documentclass[11pt]{article}

\usepackage{amsmath}
\usepackage{amssymb}
\usepackage{hyperref}
\usepackage{ulem}
\usepackage[margin=0.5in]{geometry}
\begin{document}
\maketitle

\textbf{All members of the group should work on all parts of the assignment. Copying across groups or from other sources is not allowed. We will adopt a \textbf{zero-tolerance policy} against any violation.}
\\
\\
\textbf{Submission instructions:} 
\begin{enumerate}
\item You should type out a report containing all the answers to the written problems in Word (with the equation editor) or using Latex, or write it neatly on paper and scan it. In either case, prepare a single pdf file. 
\item Put the pdf file and the code for the programming parts all in one zip file. The pdf should contain the names and ID numbers of all students in the group within the header. The pdf file should also contain instructions for running your code. Name the zip file as follows: A1-IdNumberOfFirstStudent-IdNumberOfSecondStudent.zip. (If you are doing the assignment alone, the name of the zip file is A1-IdNumber.zip). 
\item Upload the file on moodle BEFORE 11:55 pm on the due date (i.e. 3rd September). We will nevertheless allow and not penalize any submission until 10:00 am on the following day (i.e. 4th September). No assignments will be accepted thereafter. 
\item Note that only one student per group should upload their work on moodle, though all group members will receive grades. 
\item Please preserve a copy of all your work until the end of the semester. 
\end{enumerate}

\textbf{Questions:}
\begin{enumerate}
\item Consider $n$ people each of whom owns a book. The book belonging to each of $n$ persons is put into a basket. The people then pick up a book at random, due to which it is equally likely that a given person could pick any one of the $n$ books from the basket. What is the probability that
\begin{enumerate}
\item every person picks up his or her book back?
\item the first $m < n$ persons who picked up a book receive their own book back again?
\item each person among the first $m$ persons to pick up the book gets back a book belonging to one of the last $m$ persons to pick up the books?
\item Now suppose that every book put into the box has an independent probability $p$ of getting unclean, i.e. this is independent of who picked up which book and independent of whether other books became unclean. What is the probability that the first $m$ persons will pick up clean books? 
\item Continuing from the previous point, what was the probability that exactly $m$ persons will pick up clean books? \textsf{[3 $\times$ 5 = 15 points]}
\end{enumerate}

\item Given $n$ distinct values $\{x_i\}_{i=1}^n$ with mean $\mu$ and standard deviation $\sigma$, prove that for all $i$, we have $|x_i - \mu| \leq \sigma \sqrt{n-1}$. How does this inequality compare with Chebyshev's inequality as $n$ increases? (give an informal answer) \textsf{[7+3=10 points]}

\item Given $n$ values $\{x_i\}_{i=1}^n$ having mean $\mu$, median $\tau$ and standard deviation $\sigma$, prove that $|\mu-\tau| \leq \sigma$. Assume $n$ is even. \textsf{[10 points]}

\item In a certain town, there exist 100 rickshaws out of which 1 is red and 99 are blue. A person XYZ observes a serious accident caused by a rickshaw at night and remembers that the rickshaw was red in color. Hence, the police arrest the driver of the red rickshaw. The driver pleads innocence. Now, a lawyer decides to defend the hapless rickshaw driver in court. The lawyer ropes in an opthalmologist to test XYZ's ability to differentiate between the colors red and blue, under illumination conditions similar to those that existed that fateful night. The opthalmologist suggests that XYZ sees red objects as red 99\% of the time and blue objects as red 2\% of the time. What will be the main argument of the defense lawyer? (In other words, what is the probability that the rickshaw was really a red one, when XYZ observed it to be red?) \textsf{[10 points]}

\item A contestant is on a game show and is allowed to choose between three doors. Behind one of them lies a car, behind the other two there lies a stone. The contestant will be given whatever is behind the door that (s)he picked, and quite naturally (s)he wants the car. Suppose (s)he chooses the first door, and the host of the show who knows what is behind every door, opens (say) the third door, behind which there lies a stone (without opening the first door). The host now asks the contestant whether (s)he wishes to choose the second door instead of the first one. Your task is to determine whether switching the contestant's choice is going to increase his/her chance of winning the car. Remember that the host is intelligent: (s)he is always going to open a door not chosen by the contestant, \emph{and} is also going to open a door behind which there is a stone. You should approach this problem only from the point of view of conditional probability as follows. To this end, let $C_1, C_2, C_3$ be events that the car is behind doors 1,2,3 respectively. Assume $P(C_i) = 1/3, i \in \{1,2,3\}$. 
\begin{enumerate}
\item Let $Z_1$ be the event that the contestant chose door 1. Write down the value of $P(C_i|Z_1)$ for all $i \in \{1,2,3\}$.
\item Let $H_3$ be the event that the host opened door 3. Write down the value of $P(H_3|C_i,Z_1)$ for all $i \in \{1,2,3\}$.
\item Clearly the conditional probability of winning by switching is $P(C_2|H3,Z1)$. This is equal to \\ $\dfrac{P(H_3|C_2,Z_1)P(C2,Z1)}{P(H_3,Z_1)}$. Evaluate this probability. Note that $P(A_1,A_2)$ denotes the joint probability of events $A_1, A_2$. 
\item Likewise evaluate $P(C_1|H3,Z1)$.
\item Conclude whether switching is indeed beneficial. 
\item Now let us suppose that the host were quite whimsical and decided to open one of the two doors not chosen by the contestant, with equal probability, not caring whether there was a car behind the door. In this case, repeat your calculations and determine whether or not it is beneficial for the contestant to switch choices. \textsf{[2+2+5+5+1+5=20 points]}
\end{enumerate}

\textit{In the following problems, you can use the mean, median and standard deviation functions from MATLAB.}

\item Generate a sine wave in MATLAB of the form $y = 5 \sin (1.8x + \pi/3)$ where $x$ ranges from -3 to 3 in steps of 0.02. Now randomly select a fraction $f = 30\%$ of the values in the array $y$ (using MATLAB function `randperm') and corrupt them by adding random values from 100 to 120 using the MATLAB function `rand'. This will generate a corrupted sine wave which we will denote as $z$. Now your job is to filter $z$ using the following steps. 
\begin{itemize}
\item Create a new array $y_{median}$ to store the filtered sine wave.
\item For a value at index $i$ in $z$, consider a neighborhood $N(i)$ consisting of $z(i)$, 8 values to its right and 8 values to its left. For indices near the left or right end of the array, you may not have 8 neighbors in one of the directions. In such a case, the neighborhood will contain fewer values.
\item Set $y_{median}(i)$ to the median of all the values in $N(i)$. Repeat this for every $i$. 
\end{itemize}
This process is called as `moving median filtering', and will produce a filtered signal in the end. Repeat the entire procedure described here using the arithmetic mean instead of the median. This is called as `moving average filtering'. Repeat the entire procedure described here using the first quaritle (25 percentile) instead of the median. This is called as `moving quartile filtering'. Plot the original (i.e. clean) sine wave $y$, the corrupted sine wave $z$ and the filtered sine wave using each of the three methods on the same figure in different colors. Introduce a legend on the plot (find out how to do this in MATLAB). Include an image of the plot in your report. Now compute and print the relative mean squared error between each result and the original clean sine wave. The relative mean squared error between $y$ and its estimate $\hat{y}$ (i.e. the filtered signal - by any one of the different methods) is defined as $\dfrac{\sum_i (y_i-\hat{y}_i)^2}{\sum_i y^2_i}$. \\
Now repeat all the steps above using $f = 60\%$, and include the plot of the sine waves in your report, and write down the relative mean square error values. \\
Which of these methods (median/quartile/arithmetic mean) produced better relative mean squared error? Why? Explain in your report. \textsf{[5+5+4+3+3=20 points]}

\item Suppose that you have computed the mean, median and standard deviation of a set of $n$ numbers stored in array $A$ where $n$ is very large. Now, you decide to add another number to $A$. Write a MATLAB function to update the previously computed mean, another MATLAB function to update the previously computed median, and yet another MATLAB function to update the previously computed standard deviation. Note that you are \emph{not} allowed to simply recompute the mean, median or standard deviation by looping through all the data. You may need to derive formulae for this. Include the formulae and their derivation in your report. Note that your MATLAB functions should be of the following form
\begin{verbatim}
function newMean = UpdateMean (OldMean, NewDataValue, n),  
function newMedian = UpdateMedian (oldMedian, NewDataValue, A, n), 
function newStd = UpdateStd (OldMean, OldStd, NewMean, NewDataValue, n).\end{verbatim} 
Also explain, how would you update the histogram of $A$, if you received a new value to be added to $A$? (Only explain, no need to write code.)
\textbf{Note:} For updating the median, you may assume that the array $A$ is sorted in ascending order, that the numbers are all unique. For sorted arrays with a even number of elements, MATLAB returns the answer as $(A(N/2)+A(N/2+1))/2$. You may use MATLAB's convention though it is not strictly required. Recall that the standard deviation with $n$ values $A_1,...,A_n$ is given as $s_n = \sqrt{\sum_{i=1}^n (A_i-\bar{A}_n)^2/(n-1)}$ and $\bar{A}_n = \sum_{i=1}^n A_i / n $. 
\textsf{[4+5+5+1 = 15 points]}

\end{enumerate}
\end{document}